\documentclass{book}
\usepackage[polish]{babel}
\usepackage[T1]{fontenc}
\usepackage[utf8]{inputenc}
\usepackage{amsthm}
\usepackage{thmtools}

\declaretheoremstyle[headindent=15pt, headfont=\scshape\bfseries]{theoremstyle}
\declaretheorem[name=Twierdzenie, numbered=no, style=theoremstyle]{twierdzenie}

\setcounter{chapter}{11}
\setcounter{section}{6}

\begin{document}
Można jednak wówczas stosować mocniejsze kryteria logarytmiczne, do których prowadzą szeregi rozpatrywane w przykładzie 3.

Dodajmy, że istnieją szeregi (o wyrazach dodatnich), które na żadne kryterium logarytmiczne nie reagują.

\section{Szeregi Fouriera.}

Niech dany będzie szereg trygonometryczny zbieżny następującej postaci:

\[
f(x) = \frac{1}{2} a_{0} + (a_1 \cos x + b_1 \sin x) + (a_2 \cos 2x + b_2 \sin 2x) + \ldots + (a_n \cos nx + b_n \sin nx) + \ldots
\]

Zauważmy, że jeśli szereg rozważany jest jednostajnie zbieżny w przedziale $-\pi \leq x \leq \pi$, to współczynniki $a_n$ i $b_n$ dają się łatwo wyrazić w zależności od sumy szeregu, tj. od funkcji $f$. Mianowicie, ze względu na jednostajną zbieżność mamy (por. tw. 12, PARAGRAPH)

\[
\int\limits_{-\pi}^{\pi} f(x)\,dx = \int\limits_{-\pi}^{\pi} \frac{1}{2} a_0 \,dx + \sum_{n = 1}^{\infty} \int\limits_{-\pi}^{\pi} (a_n \cos nx + b_n \sin nx) \,dx
\]

Ponieważ zaś

\[
\int\limits_{-\pi}^{\pi} \cos nx \,dx = 0 = \int\limits_{-\pi}^{\pi} \sin nx \,dx, 
\int\limits_{-\pi}^{\pi} \frac{1}{2} a_0 \,dx = \pi a_0
\]

więc

\[
a_0 = \frac{1}{\pi} \int\limits_{-\pi}^{\pi} f(x) \,dx
\]

Aby obliczyć $a_0$, mnożymy obie strony równości (REF) przez $\cos nx$ i podobnie jak poprzednio znajdujemy

\[
\int\limits_{-\pi}^{\pi} f(x) \cos nx \,dx = \int\limits_{-\pi}^{\pi} \frac{1}{2} a_0 \cos nx \,dx + \sum_{m = 1}^{\infty} \int\limits_{-\pi}^{\pi} (a_m \cos mx \cos nx + b_m \sin mx \cos nx) \,dx
\]

Ponieważ zaś (por. przykład PARAGR)

\[
\int\limits_{-\pi}^{\pi} \cos^2 nx \,dx = \pi, 
\int\limits_{-\pi}^{\pi} \cos mx \cos nx \,dx = 0 = \int\limits_{-\pi}^{\pi} \sin mx \sin nx \,dx
\]

dla $m \neq n$, przeto

\[
a_n = \frac{1}{\pi} \int\limits_{-\pi}^{\pi} f(x) \cos nx \,dx
\]

Podobnie znajdujemy

\[
b_n = \frac{1}{\pi} \int\limits_{-\pi}^{\pi} f(x) \sin nx \,dx
\]

Powstaje następujące zagadnienie: jakie funkcje dadzą się przedstawić w postaci (ROWN), przy czym współczynniki $a_n$ i $b_n$ mają spełniać wzory ROWN (tzw. \textit{wzory Eulera-Fouriera}). Jeśli tego rodzaju rowinięcie istnieje, to nazywamy je \textit{szeregiem Fouriera funkcji}.

Jest rzeczą oczywistą, że ze względu na okresowość funkcji cosinus i sinus założyć należy okresowość funkcji $f$. Założymy ponadto, że funkcja $f$ jest przedziałami monotoniczna (por PAR).

Udowodnimy mianowicie następujące:

\begin{twierdzenie}
Każda funkcja okresowa $f$ o okresie $2\pi$ (tzn. $f(x + 2\pi) = f(x)$), przedziałami ciągła (wraz ze swą pochodną), przedziałami monotoniczna i spełniająca (w punktach nieciągłości) warunek
\[
f(x) = \frac{f(x - 0) + f(x + 0)}{2}
\]
daje się rozwinąć w szereg Fouriera.
\end{twierdzenie}

Oznaczmy przez $S_n (x)$ $n$-tą sumę częściową szeregu ROWN, tj.

\[
S_n(x) = \frac{1}{2} a_{0} + (a_1 \cos x + b_1 \sin x) + \ldots + (a_n \cos nx + b_n \sin nx)
\]

Należy dowieść, że jeśli współczynniki spełniają warunki (ROWN), to 

\[
f(x) = \lim_{n = \infty} S_n(x)
\]

Wymienione warunki pozwalają przekształcić wzór ROWN, jak następuje

\[
\pi S_n(x) = \inf\limits_{-\pi}^{\pi} \frac{1}{2} f(t) \,dt + 
\int\limits_{-\pi}^{\pi} f(t)(\cos t \cos x + \sin t \sin x) \,dt + 
\ldots + 
\int\limits_{-\pi}^{\pi} f(t)(\cos nt \cos nx + \sin nt \sin nx) \,dt =
\int\limits_{-\pi}^{\pi} f(t)\left(\frac{1}{2} + \cos (t -x) + \ldots + \cos n(t - x)\right) \,dt =
\int\limits_{-\pi}^{\pi} f(t) \frac{\sin (2n + 1) \frac{1}{2} (t - x)}{2 \sin \frac{1}{2} (t - x)} \,dt
\]

na mocy znanego wzoru (PAR):

\[
\frac{1}{2} + \cos t + \cos 2t + \ldots + \cos nt = \frac{\sin \frac{1}{2} (2n + 1) t}{2 \sin \frac{1}{2} t}
\]

\end{document}