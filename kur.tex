\documentclass[leqno]{book}
\usepackage[polish]{babel}
\usepackage[T1]{fontenc}
\usepackage[utf8]{inputenc}
\usepackage{amsthm}
\usepackage{thmtools}
\usepackage{amsmath}
\usepackage{scrlayer-scrpage}
\usepackage{scrextend}
\usepackage{titleref}
\usepackage{titlesec}
\usepackage{nameref}
\usepackage{hyperref}
\usepackage{relsize}

\addtolength{\voffset}{-0.75cm}
\addtolength{\textheight}{1.5cm}

\pagestyle{scrheadings}
\addtokomafont{pagehead}{\normalfont}
\chead{IV. Rachunek całkowy jednej zmiennej}
\rehead{}
\lohead{}
\cohead{\thesection.~\nameref{currentSection}}

\expandafter\def\expandafter\normalsize\expandafter{%
\normalsize
\setlength\abovedisplayskip{5pt}
\setlength\belowdisplayskip{5pt}
\setlength\abovedisplayshortskip{5pt}
\setlength\belowdisplayshortskip{5pt}
}

\declaretheoremstyle[headindent=15pt, headfont=\scshape, bodyfont=\slshape]{theoremstyle}
\declaretheorem[name=Twierdzenie, numbered=no, style=theoremstyle]{twierdzenie}

\deffootnotemark{(\textsuperscript{\thefootnotemark})}
\deffootnote[4em]{0em}{0em}{(\textsuperscript{\thefootnotemark})}
\renewcommand{\footnoterule}{%
\kern -3pt
\hrule width 5em height 0.5pt
\kern 2ex
}

\titleformat{\section}[runin]{\normalfont\bfseries}{\kern2em\thesection.}{0.5em}{}[\kern-0.5em]
\newcommand{\extendedSection}[2]{\section[#1]{#1\normalfont{\protect\footnote{#2}}\textbf{.}}\label{currentSection}}

\newcommand{\normalsign}[1]{{\normalfont{#1}}}
\newcommand{\parsign}[0]{\S\,}

\setcounter{chapter}{11}
\setcounter{section}{6}
\setcounter{page}{226}
\setcounter{equation}{28}
\counterwithout{equation}{section}

\begin{document}
    Można jednak wówczas stosować mocniejsze kryteria logarytmiczne, do których prowadzą szeregi rozpatrywane w przykładzie 3.

    Dodajmy, że istnieją szeregi (o wyrazach dodatnich), które na żadne kryterium logarytmiczne nie reagują.

    \extendedSection{Szeregi Fouriera}{J. Fourier (1768-1830). Rozwinięcia funkcji w szeregi znane pod jego imieniem wprowadził Fourier w związku z
    pracami w teorii ciepła.}

    Niech dany będzie szereg trygonometryczny zbieżny następującej postaci:

    \begin{multline}
        \label{eq:fourier}
        f(x) = \tfrac{1}{2} a_{0} + (a_1 \cos x + b_1 \sin x) + (a_2 \cos 2x + b_2 \sin 2x) + \\
        \ldots + (a_n \cos nx + b_n \sin nx) + \ldots
    \end{multline}

    Zauważmy, że jeśli szereg rozważany jest jednostajnie zbieżny w przedziale $-\pi \leq x \leq \pi$,
    to współczynniki $a_n$ i $b_n$ dają się łatwo wyrazić w zależności od sumy szeregu, tj. od funkcji $f$.
    Mianowicie, ze względu na jednostajną zbieżność mamy (por. tw. 12, \parsign10.2)

    \[
        \mathsmaller{\int\limits_{-\pi}^{\pi}} f(x)\,dx = \\ \mathsmaller{\int\limits_{-\pi}^{\pi}} \tfrac{1}{2} a_0 \,dx +
        \sum_{n = 1}^{\infty} \mathsmaller{\int\limits_{-\pi}^{\pi}} (a_n \cos nx + b_n \sin nx) \,dx \text{.}
    \]

    Ponieważ zaś

    \[
        \mathsmaller{\int\limits_{-\pi}^{\pi}} \cos nx \,dx = 0 = \mathsmaller{\int\limits_{-\pi}^{\pi}} \sin nx \,dx, \qquad
        \mathsmaller{\int\limits_{-\pi}^{\pi}} \tfrac{1}{2} a_0 \,dx = \pi a_0 \text{,}
    \]

    więc

    \begin{equation}
        \label{eq:el0}
        a_0 = \frac{1}{\pi} \mathsmaller{\int\limits_{-\pi}^{\pi}} f(x) \,dx \text{.}
    \end{equation}

    Aby obliczyć $a_0$, mnożymy obie strony równości~\eqref{eq:fourier} przez $\cos nx$ i podobnie jak poprzednio znajdujemy

    \begin{multline*}
        \mathsmaller{\int\limits_{-\pi}^{\pi}} f(x) \cos nx \,dx = \\ \mathsmaller{\int\limits_{-\pi}^{\pi}} \tfrac{1}{2} a_0 \cos nx \,dx +
        \sum_{m = 1}^{\infty} \mathsmaller{\int\limits_{-\pi}^{\pi}} (a_m \cos mx \cos nx + b_m \sin mx \cos nx) \,dx \text{.}
    \end{multline*}

    Ponieważ zaś (por. przykład 4, \parsign 10.1)

    \[
        \mathsmaller{\int\limits_{-\pi}^{\pi}} \cos^2 nx \,dx = \pi, \qquad
        \mathsmaller{\int\limits_{-\pi}^{\pi}} \cos mx \cos nx \,dx = 0 = \mathsmaller{\int\limits_{-\pi}^{\pi}} \sin mx \sin nx \,dx
    \]

    dla $m \neq n$, przeto

    \begin{equation}
        \label{eq:ela}
        a_n = \frac{1}{\pi} \mathsmaller{\int\limits_{-\pi}^{\pi}} f(x) \cos nx \,dx \text{.}
    \end{equation}

    Podobnie znajdujemy

    \begin{equation}
        \label{eq:elb}
        b_n = \frac{1}{\pi} \mathsmaller{\int\limits_{-\pi}^{\pi}} f(x) \sin nx \,dx \text{.}
    \end{equation}

    Powstaje następujące zagadnienie: jakie funkcje dadzą się przedstawić w postaci~\eqref{eq:fourier}, przy czym współczynniki $a_n$ i $b_n$ mają
    spełniać wzory~\eqref{eq:el0}-\eqref{eq:elb} (tzw. \textit{wzory Eulera-Fouriera}).
    Jeśli tego rodzaju rowinięcie istnieje, to nazywamy je \textit{szeregiem Fouriera funkcji}.

    Jest rzeczą oczywistą, że ze względu na okresowość funkcji cosinus i sinus założyć należy okresowość funkcji $f$.
    Założymy ponadto, że funkcja $f$ jest przedziałami monotoniczna (por. \parsign 4.2).

    Udowodnimy mianowicie następujące:

    \begin{twierdzenie}
        Każda funkcja okresowa $f$ o okresie $2\pi$ \normalsign{(}tzn. $f(x + 2\pi) = f(x)$\normalsign{)}, przedziałami ciągła
        \normalsign{(}wraz ze swą pochodną\normalsign{)}, przedziałami monotoniczna i spełniająca
        \normalsign{(}w punktach nieciągłości\normalsign{)} warunek
        \begin{equation}
            \label{eq:statement}
            f(x) = \frac{f(x - 0) + f(x + 0)}{2}
        \end{equation}
        daje się rozwinąć w szereg Fouriera.
    \end{twierdzenie}

    Oznaczmy przez $S_n (x)$ $n$-tą sumę częściową szeregu~\eqref{eq:fourier}, tj.

    \begin{equation}
        \label{eq:sum}
        S_n(x) = \tfrac{1}{2} a_{0} + (a_1 \cos x + b_1 \sin x) + \ldots + (a_n \cos nx + b_n \sin nx)
    \end{equation}

    Należy dowieść, że jeśli współczynniki spełniają warunki~\eqref{eq:el0}-\eqref{eq:elb}, to

    \begin{equation}
        \label{eq:fS}
        f(x) = \lim_{n = \infty} S_n(x) \text{.}
    \end{equation}

    Wymienione warunki pozwalają przekształcić wzór~\eqref{eq:sum}, jak następuje

    \begin{gather}
        \begin{align}
            \pi S_n(x) & = \mathsmaller{\int\limits_{-\pi}^{\pi}} \tfrac{1}{2} f(t) \,dt + \mathsmaller{\int\limits_{-\pi}^{\pi}} f(t)(\cos t \cos x + \sin t \sin x) \,dt +
            \ldots + \\
            & \kern 6pc + \mathsmaller{\int\limits_{-\pi}^{\pi}} f(t)(\cos nt \cos nx + \sin nt \sin nx) \,dt = \notag\\
            & = \mathsmaller{\int\limits_{-\pi}^{\pi}} f(t)\left(\tfrac{1}{2} + \cos (t -x) + \ldots + \cos n(t - x)\right) \,dt = \notag\\
            & = \mathsmaller{\int\limits_{-\pi}^{\pi}} f(t) \frac{\sin (2n + 1) \frac{1}{2} (t - x)}{2 \sin \frac{1}{2} (t - x)} \,dt \notag
        \end{align}
        \raisetag{3\baselineskip}
    \end{gather}

    na mocy znanego wzoru (por. (2), \parsign 1.2):

    \[
        \tfrac{1}{2} + \cos t + \cos 2t + \ldots + \cos nt = \frac{\sin \frac{1}{2} (2n + 1) t}{2 \sin \frac{1}{2} t}
    \]

\end{document}